\documentclass{article}
\usepackage[utf8]{inputenc}
\usepackage{tgpagella}

\usepackage[
backend=biber,
sorting=ynt,
style=alphabetic
]{biblatex}

\usepackage{hyperref}
\usepackage{amssymb,amsmath}
\usepackage{eulervm}

\addbibresource{proposal.bib}

\title{Fully Homomorphic Encryption}
\author{Ryan Doenges, Thomas Sixuan Lou}
\date{April 10th, 2017}

\begin{document}

\maketitle

\section{Overview}
We plan to implement a fully homomorphic encryption scheme. If that is too much work for one quarter, there are two potential stopping points which are cryptographically interesting in their own right: the NTRUEncrypt public key system and a leveled homomorphic scheme based on it.

Craig Gentry gave the first example of a fully homomorphic cryptosystem by \emph{bootstrapping} a leveled homomorphic scheme \cite{DBLP:conf/stoc/Gentry09}. The Gentry system involves ideal lattices generated by elements of a quotient of \(\mathbb{Z}[x]\) by a monic polynomial. This cryptographic scheme was inspired by NTRU-like cryptosystems. However, the bootstrapping approach described in this early work is far too inefficient to use in a practical implementation. 

Since the course textbook provides a leveled homomorphic scheme based on NTRU, we'll start by implementing that instead. However, the book scheme cannot be used for arbitrary computations. Our approach will be based on the scheme introduced by Alt-L\'opez, Tromer, and Vaikuntanathan (ATV) in \cite{DBLP:conf/stoc/2012}. It provides fully homomorphic encryption by bootstrapping from a leveled homomorphic scheme based on NTRU, but it's different from the leveled scheme described in the book. Its bootstrapping method is also less expensive than the early Gentry approach.

A simplified version of the ATV scheme, described by D\"oroz et al. in \cite{DBLP:journals/iacr/DorozHS14}, is available to guide our implementation. The D\"oroz paper includes a number of optimizations which we may or may not have time to implement.

\section{Roadmap}

We have a scheduled meeting once a week on Thursdays. If necessary we'll meet more frequently.

\begin{enumerate}
\item A basic testing framework \dotfill April 10th \\
\emph{Ryan}

Ryan will get the build system and testing infrastructure set up. The project is going to be implemented in OCaml \cite{ocaml}, a functional programming language with good support for interfacing with C code. There are open source tools like OUnit \cite{ounit} available for testing OCaml projects.

\item Euclidean algorithm etc. for convolution polynomial rings \dotfill April 10th\\
\emph{Thomas}

The NTRUEncrypt algorithm is built using arithmetic in the convolution polynomial rings \(R = \mathbb{Z}[x]/(x^N - 1)\) and \(R_p = \mathbb{Z}/p\mathbb{Z}[x]/(x^N - 1)\). We have implemented a simple version of \(R\). Thomas will implement \(R_p\), a \(\operatorname{gcd}\) algorithm for the rings, and any other operations we need for NTRU.

\item NTRUEncrypt public key generation \dotfill April 13th \\
\emph{Ryan}

The two most complex parts of NTRU are its public key generation and encryption algorithms. Ryan will implement key generation following the description in the book.

\item NTRUEncrypt encryption and decryption \dotfill April 13th \\
\emph{Thomas}

Thomas will implement the encryption and decryption operations as described in the textbook.

\item The book's leveled homomorphic scheme \dotfill April 20th \\
\emph{Ryan} 

The book gives a brief description of a leveled homomorphic encryption scheme. Its implementation only requires a few additional operations on top of NTRU, but it requires a very large private key for relatively few multiplication operations. Ryan will implement this using the NTRUEncrypt implementation built by Thomas.

\item The ATV leveled homomorphic scheme \dotfill May 4th \\
\emph{Ryan \& Thomas}

Provided the leveled homomorphic scheme in the book is done, we'll divide up work on the ATV scheme and try to implement it in a few weeks. As mentioned above, we're going to use the simplified version of ATV described in \cite{DBLP:journals/iacr/DorozHS14}.

\item Draft of rough draft \dotfill May 4th \\
\emph{Ryan}

Ryan will write a draft of the project write-up explaining what work has been done up to the 4th. Thomas will write sections describing his implementation work. This will all be done by the 4th so that the rough draft turned in on the 8th isn't too rough.

\pagebreak
\item The ATV fully homomorphic scheme \dotfill June 1st \\
\emph{Ryan \& Thomas}

In theory this will mean implementing the \(\mathsf{Relinearization}\) operation from \cite{DBLP:journals/iacr/DorozHS14}. In practice, we'll probably also have to implement a number of optimizations from the same paper in order to get the fully homomorphic scheme to work.
\end{enumerate}
Our timetable is pretty tight. We'll consider a feature implemented when there is code in our shared GitHub repository with tests of its functionality. The one exception to this is the testing framework, which is really a precondition to anything else being truly done since we cannot test anything until we know how we are going to write tests. 

\section{Coding experience}
Ryan has TAed the undergraduate course on functional programming and programming languages and has held industry internships as a software engineer. He currently writes programs using the Coq proof assistant as part of his research at the Programming Languages and Software Engineering (PLSE) lab.

Thomas is TAing the undergraduate programming language course this quarter. Although he is new to programming in OCaml, he's familiar with other functional languages (Haskell and SML) that share similar features.
\printbibliography

\end{document}
