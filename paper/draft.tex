\documentclass[10pt]{article}
\usepackage[utf8]{inputenc}
\usepackage{tgpagella,mathpazo}

\usepackage[
backend=biber,
sorting=ynt,
style=alphabetic
]{biblatex}

\usepackage{tikz}
\usetikzlibrary{cd}

\usepackage{dirtree}
\usepackage{hyperref}
\usepackage{amssymb,amsmath,amsthm,mathtools}

\DeclarePairedDelimiter{\round}{\lfloor}{\rceil}

\DeclareMathOperator{\Z}{\mathbb{Z}}

\addbibresource{ref.bib}

\title{Homomorphic Encryption with Rings}
\author{Ryan Doenges \and Thomas Sixuan Lou}
\date{June 8th, 2017}

\begin{document}

\maketitle

We have implemented a family of leveled homomorphic encryption schemes based on
NTRU following the textbook's exposition \cite{Hoffstein:2014:IMC:2682593} and
several research papers \cite{DBLP:conf/stoc/Lopez-AltTV12}
\cite{DBLP:journals/iacr/DorozHS14}.

\section{Introduction}

Encrypting data hides structure and patterns present in the plaintext to produce
a high-entropy ciphertext. In general, this security property means that
untrusted parties can only help store and transmit information. However, there
are cryptosystems that allow untrusted parties to perform more operations. In
particular, homomorphic encryption makes it possible for untrusted parties to
perform arbitrary computations on encrypted data while maintaining the security
guarantees provided by ordinary encryption schemes.

We have developed a framework for implementing NTRU and NTRU-like public key
cryptography and have used it to realize vanilla NTRU as well as a ``leveled
homorphic'' cryptosystem similar to NTRU. Informally, a leveled homomorphic
scheme makes it possible to evaluate circuits of bounded depth on encrypted data
without ever decrypting the data. NTRU-like schemes encrypt polynomials in rings
like \(R = \mathbb{Z}[x]/\langle \phi \rangle\), where \(\phi\) is some fixed
monic polynomial. For such a scheme to be homomorphic, it suffices for the
arithmetic operations of \(R\) or some related ring (e.g., \(R/nR\)) to commute
with the scheme's encryption and decryption maps. This is because arbitrary
binary circuits may be evaluated using the ring operations. In other words,
NTRU-like cryptosystems are homomorphic when their encryption maps \(e_k :
\mathcal{M} \to \mathcal{C}\) are ring homomorphisms.

\section{A leveled homomorphic scheme based on NTRU}

Most fully homomorphic encryption schemes are not presented as one unified
scheme. Instead, following the pioneering work of Gentry
\cite{DBLP:conf/stoc/Gentry09}, cryptographers begin with a leveled homomorphic
scheme which is then turned into a fully homomorphic scheme using a
bootstrapping process. This bootstrapping process requires the depth of the
circuit used to decrypt a ciphertext to grow polynomially with their level. We
have implemented a leveled homomorphic scheme for which the size of the
ciphertexts grows exponentially with the level, so it's not bootstrappable.
However, with small (read: insecure!) parameters we can still evaluate deep
circuits.

Let $\alpha$ and $\mu$ be fixed positive integers. The scheme, called
\emph{LHE}, is parameterized by integers $(N, p, q, d)$ in the same way as NTRU,
but with \(q > 2\alpha(2pN)^{2\mu}\). It will be able to evaluate \(\mu\)
multiplications and \(\alpha\) additions at each level. Instead of generating
private keys as arbitrary ternary polynomials in the way that ordinary NTRU
does, we generate \(f\) and \(g\) such that \(f \equiv 1 \pmod{p}\) and \(g
\equiv 0 \pmod{p}\). We additionally suppose that \(f\), \(g\), \(m\), and \(r\)
are \(p\)-bounded.

Encryption of plaintexts $m$ is given by \(e \equiv h \star r + m \pmod{q}\)
where \(r\) is drawn from a distribution \(\mathcal{T}(d, d+1)\) of ternary
polynomials defined in the textbook.

Decryption of a ciphertext $c$ is given by computing the center-lift $a$
of $f \star c$ modulo $q$ and then reducing $a$ modulo $p$.

The textbook gives a proof that when \(q > 2\alpha(2pN)^{2\mu}\), this
cryptosystem can successfully decrypt ciphertexts after $\mu$ multiplications
and $\alpha$ additions with the private key $f^\mu = f \star f \star \dots \star
f$. In practice we have found that choosing \(q\) larger than the textbook's
lower bound results in about \(1.5\mu\) to \(3\mu\) successful multiplications
before decryption fails, depending on the size of the noise parameter \(d\). Try
our demo \texttt{ntrudemo.native} to see this in action.

\section{Implementation}
The cryptosystems are in the \texttt{src/} subdirectory and their tests are in
the \texttt{tests/} subdirectory. The best place to start in order to understand
our code is the \texttt{NTRUEncrypt.ml} file, which uses operations from our
\texttt{Polynomial} module to implement the NTRU cryptosystem as described in
the textbook. For instructions on building the code and the documentation HTML
files, see \texttt{README.md}.

\subsection{Modularity}
In OCaml, \texttt{File.ml} contains an implementation of a module named
\texttt{File}. If \texttt{File.mli} exists, it defines the interface through
which other modules may use \texttt{File}. We took advantage of this modularity
to provide implementations of common operations without letting other code know
too much about the content of the module.

For example, we have a module \texttt{Int} which was for most of the quarter a
thin wrapper around the built-in \texttt{int} datatype. Its interface in
\texttt{Int.mli} keeps the type \texttt{Int.t = int} \emph{opaque}, so other
modules had to operate on \texttt{Int.t} values using functions from the
\texttt{Int} module. This meant that when we switched to using the big integer
library Zarith in the implementation of \texttt{Int}, other code worked just the
same as it did before.

This same principle underlies the implementation of and interface for the
\texttt{Polynomial} module, which represents polynomials by an array of their
coefficients but does not allow other modules to access those arrays directly.

\subsection{Challenges}

We wanted to finish implementing two more things: a linearly scaling leveled
homomorphic cryptosystem (termed DHS) based on NTRU
\cite{DBLP:journals/iacr/DorozHS14} and the Discrete Ziggurat method for
sampling from the discrete Gaussian distribution to produce better-behaved noise
polynomials \cite{ziggurat}.

We have a nearly working implementation of the DHS cryptosystem in the
\texttt{Doroz} module, but it uses modulus switching and relinearization
operations to correct noise after homomorphic multiplications and we were unable
to get those working. The process of trying to debug them broke the rest of the
code. Part of the difficulty was that the parameter estimation required to make
decryption work is only done concretely in the DHS paper for realistic security
assumptions. Our code is too slow to perform key generation for the DHS scheme
in under an hour when using parameters from their paper (\(N \approx 2^{12}, q_0
\approx 2^{155}\)). I used \texttt{perf}, a profiling tool for Linux, to
diagnose the cause of the slowness. One of its most useful analyses shows how
much time your program is spending in a given function and ranks functions by
that amount of time. Internal OCaml procedures associated with the allocation
and deallocation of arrays showed up high in the list as well as the
\texttt{conv} subroutine of \texttt{Polynomial.mul}. It seems that because the
\texttt{Polynomial} module frequently allocates new arrays for intermediate
values instead of overwriting existing arrays, there's a lot of time spent
unnecessarily managing memory.

The Discrete Ziggurat posed a different problem, which was that the paper's
algorithm for partitioning the area under a Gaussian into rectangles leaves many
details unaccounted for. Implementing the algorithm based on the description in
the paper can easily result in an optimization algorithm that spins out instead
of converging. We attempted to make our implementation work by trying to read
the published code from the authors of the Discrete Ziggurat paper, but it was
opaque and we decided uniform sampling over \(B\)-bounded polynomials would be
fine instead.

\printbibliography

\end{document}
